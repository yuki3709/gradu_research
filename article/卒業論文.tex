\documentclass{tpu-sotu}
\usepackage{ascmac}
\usepackage{cite}
\usepackage[dvipdfmx]{hyperref}
\usepackage{pxjahyper}
\usepackage{listings}
\lstset{
basicstyle={\small},% 
identifierstyle={\small},% 
commentstyle={\small\ttfamily \color[rgb]{0,0,0}},% 
keywordstyle={\small\bfseries \color[rgb]{0,0,1}},% 
ndkeywordstyle={\small},% 
stringstyle={\small\ttfamily}, 
frame={tb}, 
breaklines=true, 
columns=[l]{fullflexible},% 
numbers=left,% 
xrightmargin=0zw,% 
xleftmargin=3zw,% 
numberstyle={\scriptsize},% 
stepnumber=1, 
numbersep=1zw,% 
morecomment=[l]{//}% 
}
%
% ここでタイトルの設定をします
%
% 自分の名前
\author{尾崎 裕樹}
%
% 学籍番号
\gakusekibangou{1515015}
%
% タイトル
\title{プログラミング課題の模範解答を用いた\\テストケース評価基準の自動生成}
\etitle{English Title}
%
% 日付
\date{2018年2月}
%
% 指導教員
\professor{中村 正樹 准教授}
%
% 所属
\department{電子・情報工学科}
%
%----- begin document
%
\begin{document}
%
\maketitle
\clearpage
\pagenumbering{roman}
\tableofcontents
\clearpage
\pagenumbering{arabic}
%

% - - - - - - - - - - - - - - - - - - -
%
\chapter{はじめに}
本研究では、プログラミング教育において、教員が学生にソフトウェアテストの方法を指導する際に使用できるシステムを作成する。
\section{背景}
プログラミングを学習する上では、仕様からのコーディングだけでなく、コーディングの後に行うソフトウェアテストの方法を学ぶことも重要である。適切なソフトウェアテストを行うには、適切なテストケースの設計が必要となる。そのためには、適切なテストケースを設計するための教育が求められる。その際に、学生が設計したテストケースが適切であるかの評価を自動で行うことによって、教員の負担を減らすことができる。しかし、テストケースの入力データを評価する場合には境界値以外の値は誰が設計しても同じ値になるとは限らない。そのため、教員が模範となるテストケースを用意しても、学生のテストケースとの単純な比較だけでは評価できない。
\section{目的}
そこで、文献~\cite{a1}ではテストケースに対する評価基準を用意し、その基準をどれだけパスできるか判定することによってテストケース評価の自動化が行われている。本研究では、教員の手間をさらに削減するため、このテストケース評価基準を自動生成することを目的とする。
\section{論文の構成}
  ・・・
\chapter{関連研究}
本章では本研究の関連研究~\cite{a1}について説明する。

関連研究では、学習者自身が適切なテストケースを設計できるようになるために、作成したテストケースを評価してアドバイスを行うシステムが提案されている。このシステムでは、教員によって演習問題毎にテストケースの評価基準とテストケースが不足していた場合に表示するアドバイスが記述される。演習問題毎にテストしなければならない値や入力のデータ数が異なるため、評価基準は演習問題を分析した上で、入力のデータ構造を定義してから記述される。関連研究において、学習者が作成するプログラムは、C言語でmain関数のみか、main関数と数個の関数で記述され、データのやり取りが標準入出力となるプログラムとし、プログラムの入出力が演習問題の使用に合致するかのテストを行うことを想定されている。作成されたテストケース評価システムでは、学習者がプログラムとテストケースを入力するとテストドライバの生成とテストの実行が行われる。
\chapter{テストケース評価基準の自動生成}
\section{背景}
\chapter{検証}
\section{背景}
\chapter{まとめ}
\acknowledgements
\begin{thebibliography}{1}
   \bibitem{a1} 蜂巣吉成,小林悟,吉田敦,阿草清磁: プログラミング演習におけるテストケース評価システム,コンピュータソフトウェア第34巻第4号,2017,pp.54-60
\end{thebibliography}

\end{document}
