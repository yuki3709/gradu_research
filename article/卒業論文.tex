\documentclass{tpu-sotu}
\usepackage{ascmac}
\usepackage{cite}
\usepackage[dvipdfmx]{hyperref}
\usepackage{pxjahyper}
\usepackage{listings}
\lstset{
basicstyle={\small},% 
identifierstyle={\small},% 
commentstyle={\small\ttfamily \color[rgb]{0,0,0}},% 
keywordstyle={\small\bfseries \color[rgb]{0,0,1}},% 
ndkeywordstyle={\small},% 
stringstyle={\small\ttfamily}, 
frame={tb}, 
breaklines=true, 
columns=[l]{fullflexible},% 
numbers=left,% 
xrightmargin=0zw,% 
xleftmargin=3zw,% 
numberstyle={\scriptsize},% 
stepnumber=1, 
numbersep=1zw,% 
morecomment=[l]{//}% 
}
%
% ここでタイトルの設定をします
%
% 自分の名前
\author{尾崎 裕樹}
%
% 学籍番号
\gakusekibangou{1515015}
%
% タイトル
\title{プログラミング課題の模範解答を用いた\\テストケース評価基準の自動生成}
\etitle{English Title}
%
% 日付
\date{2018年2月}
%
% 指導教員
\professor{中村 正樹 准教授}
%
% 所属
\department{電子・情報工学科}
%
%----- begin document
%
\begin{document}
%
\maketitle
\clearpage
\pagenumbering{roman}
\tableofcontents
\clearpage
\pagenumbering{arabic}
%

% - - - - - - - - - - - - - - - - - - -
%
\chapter{はじめに}
\section{背景}
プログラミングを学習する上では、仕様からのコーディングだけでなく、コーディングの後に行うソフトウェアテストの方法を学ぶことも重要である。ソフトウェアテストを行う際には、適切なテストケースの設計が必要となる。テストケースを比較するだけでは適切なテストケースであるかは確認できない。
\section{目的}
  ・・・
\section{論文の構成}
  ・・・
\chapter{準備}
\section{ソフトウェアテスト}
\section{関連研究}
  ・・・
\subsection{サブサブセクション}
  ・・・
\chapter{テストケース評価基準の自動生成}
\section{背景}
\chapter{検証}
\section{背景}
\chapter{まとめ}
\acknowledgements
\begin{thebibliography}{1}
   \bibitem{a1} 蜂巣吉成,小林悟,吉田敦,阿草清磁: プログラミング演習におけるテストケース評価システム,コンピュータソフトウェア第34巻第4号,2017,pp.54-60
\end{thebibliography}

\end{document}
